% ======================================================================
%  Model Section
% ======================================================================

\section{Model}\label{sec:model}

I use a quantitative spatial model based on \cite{monte_commuting_2018} and \cite{ahlfeldt_economics_2015}. The key idea is that workers choose both where to live and where to work, and markets clear. I simplify by ignoring goods trade across locations---the focus is on commuting, which is what a transit investment affects most directly.

\subsection{Setup}

There are $N$ locations (census tracts in Chicago). Workers choose a residence $n$ and a workplace $i$. Each location has a fixed housing supply $H_n^R$ with price $Q_n$. There's a consumption good that's freely traded and serves as numeraire.

\subsection{Preferences}

Workers have Cobb-Douglas preferences over consumption and housing:
\begin{equation}
U_{ni\omega} = \frac{b_{ni\omega} \cdot B_n}{\kappa_{ni}} \left(\frac{c}{\alpha}\right)^\alpha \left(\frac{h}{1-\alpha}\right)^{1-\alpha}
\end{equation}
Here $B_n$ is a residential amenity, $\kappa_{ni} \geq 1$ is a commuting cost from $n$ to $i$, and $b_{ni\omega}$ is an idiosyncratic taste shock. The key assumption is that these taste shocks follow a Fr\'{e}chet distribution with shape $\varepsilon > 1$. This makes the math work out nicely.

A worker in $n$ earning wage $w_i$ at workplace $i$ has indirect utility:
\begin{equation}
V_{ni\omega} = b_{ni\omega} \cdot \frac{B_n \cdot w_i}{\kappa_{ni} \cdot Q_n^{1-\alpha}}
\end{equation}
So utility goes up with wages and amenities, and down with commuting costs and housing prices.
I treat $\kappa_{ni}$ as a non-monetary commuting disutility (an iceberg wedge in utility and choice probabilities), not a direct cash deduction from wages in the budget.

\subsection{Commuting Costs}

I assume commuting costs are exponential in travel time:
\begin{equation}
\kappa_{ni}^{-\varepsilon} = \exp(-\nu \tau_{ni})
\end{equation}
where $\tau_{ni}$ is travel time in minutes and $\nu > 0$ is the key parameter to estimate.

\subsection{Location Choice and Commuting Shares}

With Fr\'{e}chet shocks, the probability a worker chooses residence $n$ and workplace $i$ is:
\begin{equation}\label{eq:lambda_ni}
\lambda_{ni} = \frac{B_n^\varepsilon \cdot Q_n^{-(1-\alpha)\varepsilon} \cdot w_i^\varepsilon \cdot \kappa_{ni}^{-\varepsilon}}{\Phi}
\end{equation}
where $\Phi$ is a normalizing constant that sums the numerator over all $(n,i)$ pairs. 

Conditional on living in $n$, the share commuting to workplace $i$ is:
\begin{equation}\label{eq:lambda_cond}
\lambda_{ni|n} = \frac{w_i^\varepsilon \cdot \kappa_{ni}^{-\varepsilon}}{\widetilde{W}_n}
\end{equation}
where $\widetilde{W}_n = \sum_s w_s^\varepsilon \kappa_{ns}^{-\varepsilon}$ measures job access from location $n$.

This gives a gravity equation: commuting flows depend on origin and destination characteristics plus bilateral travel time. I can estimate $\nu$ from this.

\subsection{Production and Wages}

Each workplace produces using labor with diminishing returns: $Y_i = A_i (L_i^M)^\gamma$ where $A_i$ is productivity and $\gamma \leq 1$. Wages equal marginal product:
\begin{equation}
w_i = \gamma A_i (L_i^M)^{\gamma-1}
\end{equation}
When $\gamma < 1$, wages fall if more workers show up.

\subsection{Market Clearing}

Total population $\bar{L}$ is fixed. Residents in $n$ are $L_n^R = \lambda_n^R \bar{L}$ where $\lambda_n^R = \sum_i \lambda_{ni}$. Employment at $i$ equals total commuters: $L_i^M = \sum_n \lambda_{ni|n} L_n^R$. Housing markets clear: $Q_n H_n^R = (1-\alpha) \bar{w}_n L_n^R$ where $\bar{w}_n = \sum_i \lambda_{ni|n} w_i$ is expected wage income of residents in $n$ (not netted by $\kappa_{ni}$ as a monetary payment).

\subsection{Welfare}

Expected utility before taste shocks realize is $\bar{U} = \Gamma((\varepsilon-1)/\varepsilon) \cdot \Phi^{1/\varepsilon}$. So welfare changes are just changes in $\Phi^{1/\varepsilon}$.

\subsection{Inversion: Recovering Fundamentals}

To do counterfactuals, I need to know the model's fundamentals. I recover these from data as follows:
\begin{enumerate}
\item \textbf{Estimate $\nu$} from a gravity regression of commuting flows on travel times with origin and destination fixed effects.
\item \textbf{Recover wages} by finding $\{w_i\}$ that make the model's predicted employment match observed employment, given commuting costs.
\item \textbf{Recover floor prices} from land market clearing: $Q_n = (1-\alpha) \bar{w}_n L_n^R / H_n^R$.
\item \textbf{Recover amenities} from residential shares: $B_n^\varepsilon \propto \lambda_n^R Q_n^{(1-\alpha)\varepsilon} / \widetilde{W}_n$.
\item \textbf{Recover productivity} from the wage equation: $A_i = w_i / (\gamma (L_i^M)^{\gamma-1})$.
\end{enumerate}

\subsection{Counterfactual: Solving in Changes}

For the counterfactual, I use ``exact hat algebra'' following \cite{dekle_global_2007}. Let hats denote ratios: $\hat{x} = x'/x$ where primes are counterfactual values. Given changes in commuting costs $\hat{\kappa}_{ni}^{-\varepsilon} = \exp(-\nu(\tau'_{ni} - \tau_{ni}))$, I solve for changes in all endogenous variables.

The key equations in changes are:
\begin{align}
\lambda'_{ni|n} &= \frac{\lambda_{ni|n} \hat{\kappa}_{ni}^{-\varepsilon} \hat{w}_i^\varepsilon}{\sum_s \lambda_{ns|n} \hat{\kappa}_{ns}^{-\varepsilon} \hat{w}_s^\varepsilon} \\
\hat{L}_i^M &= \sum_n \frac{\lambda'_{ni|n} \hat{L}_n^R L_n^R}{L_i^M} \\
\hat{w}_i &= (\hat{L}_i^M)^{\gamma-1} \\
\hat{Q}_n &= \hat{\bar{w}}_n \hat{L}_n^R
\end{align}
I iterate these until convergence, which gives me the equilibrium response to the transit expansion.

Welfare change is:
\begin{equation}
\frac{\bar{U}'}{\bar{U}} = \left( \sum_n \sum_i \lambda_{ni} \hat{\kappa}_{ni}^{-\varepsilon} \hat{Q}_n^{-(1-\alpha)\varepsilon} \hat{w}_i^\varepsilon \right)^{1/\varepsilon}
\end{equation}
