% ======================================================================
%  Counterfactual Exercise and Results
% ======================================================================

\section{Counterfactual: Red Line Extension}\label{sec:counterfactual}

\subsection{Baseline Fundamentals}

Before running the counterfactual, it's useful to summarize the baseline fundamentals recovered from the model inversion. These describe the initial equilibrium before the transit expansion.

Table~\ref{tab:baseline} summarizes the distribution of key fundamentals across Chicago's 786 tracts.

\begin{table}[h]
\centering
\caption{Baseline Fundamentals (Model Inversion)}
\label{tab:baseline}
\begin{tabular}{lccccc}
\hline
Variable & Mean & Median & SD & Min & Max \\
\hline
Wage ($w_i$) & 1.00 & 0.97 & 0.22 & 0.41 & 2.47 \\
Floor Price ($Q_n$) & 1.07 & 1.00 & 0.47 & 0.44 & 6.42 \\
Amenity ($B_n$) & 1.75 & 1.01 & 2.88 & 0.06 & 40.77 \\
Commuting Access ($\widetilde{W}_n$) & 327 & 306 & 143 & 83 & 982 \\
\hline
\end{tabular}
\end{table}

Wages are highest in the Loop and surrounding business districts, declining as you move to the periphery. Floor prices are highest along the lakefront and in the North Side. Amenities show the widest variation---some tracts have very high amenities (lakefront, parks) while others are quite low.

Figure~\ref{fig:baseline_maps} shows maps of the key baseline fundamentals. The spatial patterns are intuitive: high wages downtown, high floor prices along the lake, and amenities concentrated in desirable neighborhoods.

\begin{figure}[h]
\centering
\includegraphics[width=\textwidth]{figures/baseline_fundamentals_maps.pdf}
\caption{Baseline fundamentals from model inversion: wages, floor prices, amenities, and commuting market access by census tract.}
\label{fig:baseline_maps}
\end{figure}

\subsection{Counterfactual Setup}

To evaluate the Red Line Extension, I compute travel times under two scenarios:
\begin{enumerate}
\item \textbf{Baseline}: Current CTA network (Red Line ends at 95th Street)
\item \textbf{Extension}: CTA network with four new stations at 103rd, 111th, Michigan/116th, and 130th
\end{enumerate}

I modify the GTFS schedule to add the new stations with realistic travel times between stops, then use \texttt{r5r} to compute the full origin-destination travel time matrix under each scenario.

Figure~\ref{fig:rle_stations} shows the location of the proposed stations relative to existing transit infrastructure. The extension would bring rail service to Far South Side neighborhoods that currently rely on buses to reach the existing Red Line terminal.

\begin{figure}[h]
\centering
\includegraphics[width=0.7\textwidth]{figures/rle_stations_map.pdf}
\caption{Proposed Red Line Extension stations (red points) and existing Red Line (red line). The extension adds four stations south of the current 95th Street terminal.}
\label{fig:rle_stations}
\end{figure}

The travel time improvements are concentrated among tracts near the new stations. For origin-destination pairs involving Far South Side tracts, travel times to downtown destinations fall by 10--20 minutes in some cases. Most tract pairs see no change.

I define ``treated'' tracts as those with treatment intensity above 10\%, where intensity is based on the average travel time improvement across all destination tracts. By this measure, 24 tracts are directly treated, and about 13,800 origin-destination pairs see reduced travel times.

\subsection{Results}

Table~\ref{tab:results} summarizes the counterfactual results.

\begin{table}[h]
\centering
\caption{Counterfactual Results: Red Line Extension}
\label{tab:results}
\begin{tabular}{lcc}
\hline
& Treated Tracts & All Tracts \\
\hline
Mean $\hat{Q}$ (floor price ratio) & 1.063 & 1.000 \\
Mean $\hat{L}^R$ (population ratio) & 1.038 & 1.000 \\
Mean $\hat{w}$ (wage ratio) & 0.999 & 1.000 \\
\hline
Aggregate welfare change & \multicolumn{2}{c}{0.067\%} \\
Number of treated tracts & \multicolumn{2}{c}{24} \\
Improved OD pairs & \multicolumn{2}{c}{13,800} \\
\hline
\end{tabular}
\end{table}

\paragraph{Welfare.} The aggregate welfare gain is about 0.067\%. This is modest but meaningful given the localized nature of the intervention. For context, \cite{monte_commuting_2018} find that observed reductions in commuting costs across all U.S.\ counties generate welfare gains of around 3.3\%---much larger because their counterfactual affects the entire country.

\paragraph{Housing prices.} Treated tracts see floor prices rise by about 6.4\% on average. This reflects increased demand for housing in these areas as commuting downtown becomes easier. For non-treated tracts, the effects are negligible.

\paragraph{Population.} Treated tracts gain population, with an average increase of about 3.8\%. Workers are moving toward areas with better transit access. The population gains come at the expense of other tracts, but the effects are spread thinly across the rest of the city. Figure~\ref{fig:counterfactual_maps} shows the full spatial pattern across outcomes.

\begin{figure}[h]
\centering
\includegraphics[width=0.95\textwidth]{figures/counterfactual_maps.pdf}
\caption{Counterfactual spatial effects by tract: wage change, floor price change, residential population change, and resident welfare change under the main specification.}
\label{fig:counterfactual_maps}
\end{figure}

\paragraph{Wages.} Wage changes are very small (essentially zero). This is because the labor supply shifts involved are modest, and $\gamma = 0.65$ implies fairly elastic labor demand. With diminishing returns, wages only adjust substantially when employment changes are large.

\subsection{Interpretation}

The results tell a coherent story. The Red Line Extension improves job access for Far South Side residents, making these neighborhoods more attractive places to live. This pulls in more residents, which bids up housing prices. The welfare gains come from two sources: (1) direct utility gains from shorter commutes, and (2) improved matching between workers and jobs as some commutes become feasible that weren't before.

The 0.067\% aggregate welfare gain is small partly because only about 3\% of Chicago's tracts (24 out of 786) are directly treated. But residents of those tracts gain substantially---their housing prices rise by over 6\%, reflecting the capitalized value of better transit access.

\subsection{Caveats}

Several caveats are worth noting:

\begin{itemize}
\item I assume fundamentals (amenities, productivity) don't change. In reality, better transit might attract businesses or amenities to the corridor.
\item The model is static. I compare two steady states without modeling the transition dynamics.
\item I only consider transit commuting. Some Far South Side residents currently drive; they would see smaller benefits.
\item The welfare measure is average utility. It doesn't capture distributional effects---existing homeowners near new stations gain from rising property values, while renters may be squeezed.
\item I abstract from construction costs. A full cost-benefit analysis would need to weigh these benefits against the capital and operating costs of the extension.
\end{itemize}
