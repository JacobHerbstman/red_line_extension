% ======================================================================
%  Counterfactual Exercise and Results
% ======================================================================

\section{Counterfactual: Red Line Extension}\label{sec:counterfactual}

\subsection{Counterfactual Setup}

To evaluate the Red Line Extension, I compute travel times under two scenarios:
\begin{enumerate}
\item Baseline: Current CTA network (Red Line ends at 95th Street)
\item Extension: CTA network with four new stations at 103rd, 111th, Michigan/116th, and 130th
\end{enumerate}

I modify the GTFS schedule to add the new stations with realistic travel times between stops, then use \texttt{r5r} to compute the full origin-destination travel time matrix under each scenario.

Figure~\ref{fig:rle_stations} shows the location of the proposed stations relative to existing transit infrastructure. The extension would bring rail service to Far South Side neighborhoods that currently rely on buses to reach the existing Red Line terminal.

\begin{figure}[h]
\centering
\includegraphics[width=0.7\textwidth]{figures/rle_stations_map.pdf}
\caption{Proposed Red Line Extension stations (red points) and existing Red Line (red line). The extension adds four stations south of the current 95th Street terminal.}
\label{fig:rle_stations}
\end{figure}

The travel time improvements are concentrated among tracts near the new stations. For origin-destination pairs involving Far South Side tracts, travel times to downtown destinations fall by 10--20 minutes in some cases. Most tract pairs see no change.

I define ``treated'' tracts as those with treatment intensity above 10\%, where intensity is based on the average travel time improvement across all destination tracts. By this measure, 24 tracts are directly treated, and about 13,800 origin-destination pairs see reduced travel times.

\subsection{Results}

Table~\ref{tab:results} summarizes the counterfactual results.

\begin{table}[h]
\centering
\caption{Counterfactual Results: Red Line Extension}
\label{tab:results}
\begin{tabular}{lcc}
\hline
& Treated Tracts & All Tracts \\
\hline
Mean $\hat{Q}$ (floor price ratio) & 1.063 & 1.000 \\
Mean $\hat{L}^R$ (population ratio) & 1.038 & 1.000 \\
Mean $\hat{w}$ (wage ratio) & 0.999 & 1.000 \\
\hline
Aggregate welfare change & \multicolumn{2}{c}{0.067\%} \\
Number of treated tracts & \multicolumn{2}{c}{24} \\
Improved OD pairs & \multicolumn{2}{c}{13,800} \\
\hline
\end{tabular}
\end{table}

\paragraph{Welfare.} The aggregate welfare gain is about 0.067\%. This is modest in the aggregate, but meaningful for the areas it affects directly. For context, \cite{monte_commuting_2018} find that observed reductions in commuting costs across all U.S.\ counties generate welfare gains of around 3.3\%---much larger because their counterfactual affects the entire country and are likely much better targeted.

\paragraph{Housing prices.} Treated tracts see floor prices rise by about 6.4\% on average. This reflects increased demand for housing in these areas as commuting downtown becomes easier. For non-treated tracts, the effects are negligible.

\paragraph{Population.} Treated tracts gain population, with an average increase of about 3.8\%. Workers are moving toward areas with better transit access. The population gains come at the expense of other tracts, but the effects are spread thinly across the rest of the city. Figure~\ref{fig:counterfactual_maps} shows the full spatial pattern across outcomes.

\begin{figure}[h]
\centering
\includegraphics[width=0.95\textwidth]{figures/counterfactual_maps.pdf}
\caption{Counterfactual spatial effects by tract: wage change, floor price change, residential population change, and resident welfare change under the main specification.}
\label{fig:counterfactual_maps}
\end{figure}

\paragraph{Wages.} Wage changes are very small (essentially zero). This is because the labor supply shifts involved are modest, and $\gamma = 0.65$ implies fairly elastic labor demand. With diminishing returns, wages only adjust substantially when employment changes are large.

\subsection{Interpretation}

The Red Line Extension improves job access for Far South Side residents, making these neighborhoods more attractive places to live. This pulls in more residents, which bids up housing prices. The welfare gains come from two sources: (1) direct utility gains from shorter commutes, and (2) improved matching between workers and jobs as some commutes become feasible that weren't before.

The 0.067\% aggregate welfare gain is small partly because only about 3\% of Chicago's tracts (24 out of 786) are directly treated. But residents of those tracts gain substantially. Their housing prices rise by over 6\%, reflecting the capitalized value of better transit access.
