% ======================================================================
%  Data and Estimation
% ======================================================================

\section{Data and Estimation}\label{sec:data}

\subsection{Data Sources}

I use several data sources:

\paragraph{Commuting flows.} Bilateral commuting flows come from the LODES (LEHD Origin-Destination Employment Statistics) data. This gives workplace-residence pairs at the census block level, which I aggregate to tracts. I use the 2019 vintage to avoid pandemic distortions.

\paragraph{Travel times.} I compute transit travel times between tract centroids using the \texttt{r5r} package, which routes trips on the actual CTA network using GTFS schedule data. I assume a Wednesday 8am departure to capture typical commuting conditions. For the counterfactual, I create a modified GTFS file that adds the four new Red Line stations with realistic stop times.

\paragraph{Housing supply.} Residential floor space by tract comes from Cook County Assessor data. I aggregate building square footage for residential properties to the tract level.

\paragraph{Geography.} Census tract boundaries come from the Census Bureau's TIGER files (2010 vintage to match LODES).

\subsection{Sample Construction}

I focus on census tracts within Chicago. I drop tracts with very little residential floor space (under 100,000 sq ft) since these are mostly commercial/industrial areas where the residential choice model doesn't make sense. This leaves 786 tracts.

\subsection{Estimation of $\nu$}

The gravity equation for commuting is:
\begin{equation}
\mathbb{E}[F_{ni}] = \exp(\theta_n + \zeta_i - \nu \tau_{ni})
\end{equation}
where $F_{ni}$ is the flow from residence $n$ to workplace $i$, $\theta_n$ and $\zeta_i$ are origin and destination fixed effects, and $\tau_{ni}$ is travel time in minutes.

I estimate this by PPML (Poisson pseudo-maximum likelihood) with two-way fixed effects, following \cite{santos_silva_log_2006}. PPML handles zeros well and is consistent under heteroskedasticity.

Table~\ref{tab:gravity} shows the results. The main estimate is $\hat{\nu} = 0.039$, meaning an additional minute of commute time reduces the flow by about 3.8\%. This is somewhat lower than the estimates from \cite{ahlfeldt_economics_2015}, who find around 0.07 per minute for Berlin. The difference likely reflects that Chicago has more automobile commuting than Berlin, so transit times matter less on the margin.

\begin{table}[h]
\centering
\caption{Gravity Estimation Results}
\label{tab:gravity}
\begin{tabular}{lcc}
\hline
& PPML & OLS \\
\hline
Travel time (min) & $-0.039$ & $-0.016$ \\
& $(0.0005)$ & $(0.0004)$ \\
\hline
Origin FE & Yes & Yes \\
Destination FE & Yes & Yes \\
Observations & 481,801 & 127,912 \\
\hline
\multicolumn{3}{l}{\footnotesize Standard errors in parentheses.} \\
\multicolumn{3}{l}{\footnotesize OLS uses only positive flows with log transformation.}
\end{tabular}
\end{table}

Figure~\ref{fig:gravity_binscatter} shows the residualized binned relationship between log commuting flows and travel time. The relationship is close to linear and clearly downward sloping, consistent with the PPML estimate.

\begin{figure}[h]
\centering
\includegraphics[width=0.85\textwidth]{figures/gravity_binscatter_residualized.pdf}
\caption{Residualized gravity binscatter: log commuting flows vs. transit travel time. Both variables are residualized on origin and destination fixed effects.}
\label{fig:gravity_binscatter}
\end{figure}

The OLS estimate (on log flows, positive flows only) is smaller at 0.016. This is expected---OLS with log transformation gives inconsistent estimates under heteroskedasticity and drops zeros, which can bias the coefficient toward zero. PPML is preferred.

\subsection{Calibrated Parameters}

I set the following parameters based on the literature:
\begin{itemize}
\item $\varepsilon = 6.83$: Fr\'{e}chet shape parameter, from \cite{ahlfeldt_economics_2015}
\item $1-\alpha = 0.30$: Housing expenditure share
\item $\gamma = 0.65$: Labor share in production
\end{itemize}

\subsection{Model Inversion}

With $\nu$ estimated and other parameters calibrated, I invert the model to recover fundamentals for all 786 tracts. The inversion proceeds in steps as described in Section~\ref{sec:model}. The key step is recovering wages: I iterate on the labor market clearing condition until predicted employment matches observed employment. Then floor prices and amenities follow from the other equilibrium conditions.

The model fits the data well---the correlation between observed and model-predicted residents/workers is essentially 1.0 by construction (the inversion is designed to match these moments exactly).

The recovered wages range from about 0.4 to 2.5 times the mean (normalized to 1). Higher-wage tracts are concentrated in and around the Loop, which makes sense given that's where most high-paying jobs are located. Amenities vary more widely, with high values along the lakefront and in popular North Side neighborhoods.
