% ======================================================================
%  Abstract and Introduction
% ======================================================================

\begin{abstract}
This paper uses a quantitative spatial model to evaluate the welfare effects of the proposed Chicago Red Line Extension. The extension would add four new stations on the South Side, improving transit access for underserved neighborhoods. I build a model where workers choose where to live and work based on wages, housing costs, amenities, and commuting times. I estimate the model using commuting flow data and GTFS-based travel times, then simulate the effects of the new transit line. The results suggest aggregate welfare gains of about 0.07\%, with larger effects concentrated near the new stations where housing prices rise by around 6\% and population increases by nearly 4\%.
\end{abstract}

\section{Introduction}\label{sec:intro}

The Chicago Transit Authority has proposed extending the Red Line south from 95th Street to 130th Street, adding four new stations. This would bring rapid transit to Far South Side neighborhoods that currently lack direct rail access to downtown.

However, the project has ballooned in cost to around \$5.6 billion, which would make it one of the most expensive per mile transit projects in the world. In addition, ridership projections are unlikely to be hit, as they rely on Chicago gaining over $500,000$ residents by 2030. 

Given the high cost and uncertainty in forecasted ridership, it is important to carefully evaluate the welfare effects of the Red Line Extension. These welfare effects can come through multiple channels. First, commuters who use the new line will save time on their trips to work. Second, improved transit access can make nearby neighborhoods more attractive, raising housing prices and changing where people choose to live. Third, better access to downtown jobs could affect wages and employment opportunities for residents in other parts of the city. 

To capture these general equilibrium effects, I use a quantitative spatial model in the style of \cite{ahlfeldt_economics_2015} and \cite{monte_commuting_2018}. In the model, workers choose both where to live and where to work. They care about wages, housing costs, local amenities, and how long it takes to commute. Everyone has idiosyncratic preferences over locations, which generates a typical gravity equation for commuting flows that I estimate from data. 

I do this using PPML regression on observed commuting flows and travel times computed from CTA's GTFS schedule. I find that each additional minute of travel time reduces commuting flows by about 3.9\%. I then invert the model to back out the underlying fundamentals that are consistent with the observed data. Finally, I compute new travel times with the extension in place and solve for the new equilibrium to evaluate aggregate welfare changes.

The main finding is that the Red Line Extension generates muted positive welfare gains of about 0.07\% in aggregate. These effects are concentrated in the 24 tracts near the new stations. These tracts see housing price increases of about 6\% and population gains of nearly 4\% as improved job access makes these areas more attractive.

The rest of the paper proceeds as follows. Section~\ref{sec:background} gives background on the Red Line Extension and explains why a spatial model is appropriate. Section~\ref{sec:model} presents the model. Section~\ref{sec:data} describes the data and estimation. Section~\ref{sec:counterfactual} presents the counterfactual results. Section~\ref{sec:conclusion} concludes.
