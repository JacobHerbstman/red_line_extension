% ======================================================================
%  Abstract and Introduction
% ======================================================================

\begin{abstract}
This paper uses a quantitative spatial model to evaluate the welfare effects of the proposed Chicago Red Line Extension. The extension would add four new stations on the South Side, improving transit access for underserved neighborhoods. I build a model where workers choose where to live and work based on wages, housing costs, amenities, and commuting times. I estimate the model using commuting flow data and GTFS-based travel times, then simulate the effects of the new transit line. The results suggest aggregate welfare gains of about 0.07\%, with larger effects concentrated near the new stations where housing prices rise by around 6\% and population increases by nearly 4\%.
\end{abstract}

\section{Introduction}\label{sec:intro}

The Chicago Transit Authority has proposed extending the Red Line south from 95th Street to 130th Street, adding four new stations. This would bring rapid transit to Far South Side neighborhoods that currently lack direct rail access to downtown. But how much would this actually benefit residents? And how would the effects ripple through the local economy?

Answering these questions is tricky because transit improvements don't just save commuters time---they also change where people want to live and work. When a neighborhood gets better transit access, more people might want to move there, which pushes up housing prices. More workers commuting to downtown could affect wages there too. A simple back-of-the-envelope calculation that just counts time savings would miss all of this.

To capture these general equilibrium effects, I use a quantitative spatial model in the style of \cite{ahlfeldt_economics_2015} and \cite{monte_commuting_2018}. In the model, workers choose both where to live and where to work. They care about wages, housing costs, local amenities, and how long it takes to commute. Everyone has idiosyncratic preferences over locations, which generates a gravity equation for commuting that I can estimate from the data.

The main parameter I need to estimate is how much commuting flows respond to travel time. I do this using PPML regression on observed commuting flows and travel times computed from CTA's GTFS schedule. I find that each additional minute of travel time reduces commuting flows by about 3.9\%. I then ``invert'' the model to back out the underlying fundamentals (wages, housing prices, amenities) that are consistent with the observed data. Finally, I compute new travel times with the extension in place and solve for the new equilibrium.

The main finding is that the Red Line Extension generates positive welfare gains of about 0.07\% in aggregate. While modest citywide, the effects are substantial locally: the 24 tracts near the new stations see housing price increases of about 6\% and population gains of nearly 4\% as improved job access makes these areas more attractive.

The rest of the paper proceeds as follows. Section~\ref{sec:background} gives background on the Red Line Extension and explains why a spatial model is appropriate. Section~\ref{sec:model} presents the model. Section~\ref{sec:data} describes the data and estimation. Section~\ref{sec:counterfactual} presents the counterfactual results. Section~\ref{sec:conclusion} concludes.
