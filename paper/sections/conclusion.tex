% ======================================================================
%  Conclusion
% ======================================================================

\section{Conclusion}\label{sec:conclusion}

This paper uses a quantitative spatial model to evaluate the Chicago Red Line Extension. I estimate the model using data on commuting flows and transit travel times, then simulate the equilibrium effects of adding four new stations on the Far South Side.

The results suggest that the extension would generate quite modest aggregate welfare gains of about 0.07\%. These effects are concentrated locally: the 24 tracts near the new stations see housing prices rise by over 6\% and population increase by nearly 4\%. These local effects reflect the capitalized value of improved job access.

The small aggregate number reflects the localized nature of the intervention and the fact that only about 3\% of Chicago's tracts are directly treated. But for Far South Side residents who currently face long commutes or limited job access, the benefits would be substantial.

Several extensions would improve the analysis. Adding agglomeration externalities would let the extension have multiplier effects if it helps employment cluster more efficiently. Modeling housing supply responses would capture how developers might respond to increased demand. 

And most importantly, including multiple worker skill groups and calibrating to tract-level income distributions would allow for distributional analysis. 

Overall, due to the enormous cost of the Red Line Extension, these modest welfare gains suggest careful consideration is needed before proceeding. Targeted transit improvements that benefit a larger share of residents or better connect high-demand areas might yield larger returns.

