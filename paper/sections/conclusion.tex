% ======================================================================
%  Conclusion
% ======================================================================

\section{Conclusion}\label{sec:conclusion}

This paper uses a quantitative spatial model to evaluate the Chicago Red Line Extension. I estimate the model using data on commuting flows and transit travel times, then simulate the equilibrium effects of adding four new stations on the Far South Side.

The results suggest that the extension would generate aggregate welfare gains of about 0.07\%. While modest at the city level, the effects are concentrated locally: the 24 tracts near the new stations see housing prices rise by over 6\% and population increase by nearly 4\%. These local effects reflect the capitalized value of improved job access.

The small aggregate number reflects the localized nature of the intervention---only about 3\% of Chicago's tracts are directly treated. But for Far South Side residents who currently face long commutes or limited job access, the benefits would be substantial.

Several extensions would improve the analysis. Adding agglomeration externalities would let the extension have multiplier effects if it helps employment cluster more efficiently. Modeling housing supply responses would capture how developers might respond to increased demand. And a dynamic model could trace out the transition path rather than just comparing steady states.

For the purposes of this exercise, the main takeaway is that quantitative spatial models are useful for policy evaluation. They let us think through general equilibrium effects---how location choices, prices, and wages all adjust together---while staying disciplined by what we observe in the data.
