% ======================================================================
%  Background and Diagnostic
% ======================================================================

\section{Background}\label{sec:background}

\subsection{The Red Line Extension}

The CTA Red Line runs north-south through Chicago, currently ending at 95th/Dan Ryan on the South Side. Communities further south such as Roseland, Pullman, West Pullman, and Riverdale don't have direct rail access to downtown, even though residents there often have some of the longest commute times in the city.
The city has promised this extension since the 1960s to improve transit access for these underserved neighborhoods.

The proposed Red Line Extension would add about 5.6 miles of track and four new stations:
\begin{itemize}
    \item 103rd Street
    \item 111th Street  
    \item Michigan Avenue (near 116th Street)
    \item 130th Street (the new terminal)
\end{itemize}

This project has been in planning for a long time. The CTA estimates it would serve around 30,000 riders per day and give Far South Side residents a one-seat ride to the Loop, however these projections rely on unrealistically strong population growth to $3.2$ million people by 2030 and may be overly optimistic.

\subsection{Related Literature}

The model I use comes from \cite{ahlfeldt_economics_2015} and \cite{monte_commuting_2018}. \cite{ahlfeldt_economics_2015} study Berlin and use the division/reunification to identify agglomeration forces. They also do counterfactuals about what Berlin would look like without cars. \cite{monte_commuting_2018} look at commuting across U.S.\ counties and show that commuting linkages matter a lot for how local labor markets respond to shocks.
In addition, the transportation improvement counterfactual is in the spirit of \cite{allen_welfare_2022}, but the advanced network structure of their paper is out of the scope of this exericse.

